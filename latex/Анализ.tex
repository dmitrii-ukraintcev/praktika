\section{Анализ предметной области}
\subsection{Обзор истории CMS}
Широкая популярность Интернета обусловнена появлением Всемирной паутины в 1991 году. Изначально считалось, что привлекательность дизайна сайтов не столь важна, как предоставляемая ими информация. Это было связанно с ограниченными возможностями компьютерного оборудования. Веб-сайты были статичными и создавались вручную с использованием HTML-разметки.

С ростом мощности персональных компьютеров и появлением таких технологий как JavaScript (1995) и CSS (1996), интернет стал более наглядным и функциональным.

Параллельно развивалось серверное программирование в 90-е годы, появились языки программирования, такие как PHP (1995), Java (1995), технология Active Server Pages (1996), и система управления базами данных MySQL (1994).

В 2005 году появилась технология AJAX, позволяющая обновлять данные без перезагрузки страницы. Быстрое развитие программного обеспечения привело к разделению веб-сайтов на функциональные блоки: контент (MySQL, HTML), дизайн (CSS) и бизнес-логика (PHP, JavaScript).

В конце 90-х годов наблюдался стремительный рост интернет-контента, что побудило предприятия использовать корпоративные веб-сайты, однако их поддержка в основном выполнялась вручную программистами, затрудняя своевременную публикацию контента. Это создало потребность в системах, автоматизирующих и оптимизирующих процессы работы с контентом, таких как системы управления контентом (CMS).

Первые CMS появились в середине 1990-х годов, разрабатывались организациями самостоятельно и ориентировались на нужды конкретных компаний. В период С 1995 по 1997 годы появились системы управления корпоративным контентом, такие как FileNet, StoryBuilder, Intercloth, Documentum, FatWire, FutureTense и Inso.

С начала 2000-х годов происходит активное создание систем управления веб-контентом (WCMS). В это время утвердилось мнение о том, что современный сайт состоит из двух ключевых компонентов – дизайна и контента. Программисты отвечают за дизайн, а профессионалы в предметной области обеспечивают контент. Это способствовало привлечению большого числа участников к созданию сайтов, что привело к увеличению объема и качества информации в Интернете.

Появление CMS с открытым исходным кодом, таких как Mambo, Drupal, WordPress и Joomla, а также коммерческих CMS, таких как NetCat, Shop-Script, Битрикс: Управление сайтом 3.0 и CS-Cart, сделало создание сайтов доступным для широкого круга пользователей.

WCMS формировались вокруг четырех основных функций: создание контента с использованием редакторов WYSIWYG, управление контентом, публикация контента на сайте и презентация данных для улучшения их визуального представления.

Определились различные группы пользователей, включая дизайнеров, администраторов, команду внедрения и авторов контента. 

Уникальность WCMS обусловлена многочисленными шаблонами, плагинами и доступом к CSS.

Тем не менее, первые WCMS имели некоторые недостатки, такие как сложность конфигурации CSS, ограниченную функциональность редакторов WYSIWYG и ограниченный круг пользователей, способных создавать контент.

На дальнейшее развитие WCMS, оказали влияние несколько выжных факторов.

Во-первых, это дальнейшее развитие вычислительной техники. Мощность современных смартфонов превосходит ту, которую имел персональный компьютер 20 лет назад.

Во-вторых, появление в 2005 году концепции и технологии Web 2.0, социальных сетей и облачных вычислений. Web 2.0 расширил возможности Интернета в целом и WCMS в частности. Теперь не только отдельные люди, но и целые сообщества могли вносить свой вклад в информационные ресурсы, что привело к увеличению объема доступной информации. В результате возникла потребность в более простых инструментах для работы с контентом, таких как вики-разметка и онлайн-редакторы. Это также вызвало рост спроса на интерфейсы, ориентированные на непрофессионалов в информационных системах. Появляются новые более удобные и функциональные версии редакторов WYSIWYG, а установка и первоначальная настройка WCMS стала гораздо быстрее и проще. Рост социальных сетей требует интеграции с ними WCMS, которая происходит через плагины для автоматического связывания и регистрации через социальные сети.

Третий фактор -- быстрое развитие мобильных технологий и увеличение трафика с мобильных устройств влияет на тенденции развития веб-сайтов.

\subsection{Основные понятия CMS}
CMS -- программный комплекс, используемый для обеспечения и организации совместного процесса создания, редактирования и контроля содержимого веб-страниц. Содержимое может включать текст, изображения, видео, аудиофайлы, документы, мультимедийные файлы и многое другое.

CMS позволяет нетехническим пользователям легко управлять и обновлять контент на веб-сайте, не требуя навыков программирования или веб-разработки.

CMS, как правило, имеет модульную архитектуру, обеспечивающую легкую интеграцию плагинов и расширений, которые могут быть настроены для удовлетворения конкретных требований или расширения функциональности.

Пользователи CMS, такие как авторы или редакторы, создают контент с помощью  WYSIWYG редактора, позволяющего легко форматировать и манипулировать текстом, изображениями и мультимедийными компонентами.

Созданный контент хранится в базе данных вместе с метаданными, такими как информация об авторе, категории и теги, которые облегчают организацию и возможность поиска.

Авторизованные пользователи могут управлять содержимым, выполняя такие действия, как редактирование, просмотр, утверждение или удаление содержимого, а также управление ролями пользователей и разрешениями на доступ.

Когда пользователь запрашивает определенную страницу или ресурс, CMS извлекает соответствующий контент из базы данных, обрабатывает его, используя шаблоны и темы для стилизации, и генерирует окончательный HTML-документ, который затем передается в веб-браузер пользователя.

\subsection{Структура CMS}
Хотя платформы CMS могут различаться по функциональности, у них есть общие основные компоненты. Эти компоненты включают в себя:
\begin{itemize}
	\item приложение для управления контентом (CMA). Приложение для управления контентом (CMA) -- это пользовательский интерфейс, который позволяет создателям и редакторам контента создавать, изменять и удалять контент с веб-сайта без необходимости наличия технических знаний. Это часть CMS, которую чаще всего используют создатели контента и администраторы;
	\item приложение доставки контента (CDA). Приложение доставки контента (CDA) отвечает за хранение и доставку контента конечным пользователям. Он извлекает содержимое из базы данных, объединяет его с соответствующими шаблонами и отображает на веб-сайте. Этот процесс происходит в фоновом режиме и невидим для создателей контента и администраторов;
	\item база данных. База данных хранит и упорядочивает контент и метаданные веб-сайта. Платформы CMS обычно используют базы данных для хранения контента, шаблонов, пользовательской информации и конфигураций.
\end{itemize}

\subsection{Классификация CMS}
Существует несколько типов CMS, каждая из которых отличается архитектурой, функциональностью и вариантами использования. Выделяют три основных типа CMS:
\begin{itemize}
	\item монолитная (связанная) CMS. Монолитная или совмещенная CMS -- это традиционная система управления контентом с тесно интегрированными уровнями управления контентом и представления. Этот тип CMS поставляется со встроенными шаблонами и инструментами дизайна для создания и поддержания внешнего вида веб-сайта. Монолитные платформы CMS обычно предлагают более оптимизированный опыт для нетехнических пользователей, но они могут быть менее гибкими, чем безголовые или развязанные варианты CMS;
	\item безголовая CMS. Безголовая CMS -- это система управления контентом, которая не имеет внешнего интерфейса или уровня представления. Вместо этого контент отделен от представления, что позволяет разработчикам выбирать любую интерфейсную технологию для отображения контента. В безголовой CMS контент управляется через API (программные интерфейсы приложений), которые могут обслуживать контент на нескольких устройствах и платформах, что делает его популярным выбором для предприятий с несколькими каналами доставки, такими как веб-сайты, мобильные приложения и устройства IoT;
	\item развязанная CMS. Развязанная CMS -- это гибрид безголовой и традиционной монолитной (связанной) CMS. Как и безголовая CMS, несвязанная CMS отделяет управление контентом от уровня представления. Тем не менее, он также включает в себя встроенные интерфейсные шаблоны и инструменты, позволяющие создавать и предварительно просматривать контент перед запуском. Это позволяет создателям контента иметь больший контроль над представлением своего контента, в то же время используя преимущества гибкости и масштабируемости несвязанной архитектуры.
\end{itemize}